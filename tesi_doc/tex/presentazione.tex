\documentclass{beamer}
\usepackage[utf8]{inputenc}
\usepackage[italian]{babel}
\usepackage{physics}
\usepackage{amsmath}
\usepackage{amssymb}
\usepackage{bbm} % \mathbbm{1}
\usepackage{utopia} %font utopia imported
\usepackage{tikz}
\usepackage{graphicx}
\usepackage[font=scriptsize,labelfont=bf]{caption}
\usepackage[labelformat=empty]{subcaption}
\usepackage{wrapfig}
\usepackage{multirow}
\usepackage{color}
\usepackage{xcolor}
\usepackage{changepage}

\graphicspath{ {../pics/} }

\newcommand{\D}[1]{\,\mathcal{D}#1\,}
\newcommand{\Z}{\mathcal{Z}}
\newcommand{\R}{\mathbb{R}}
\newcommand{\SU}{\operatorname{SU}}
\newcommand{\Sp}{\operatorname{Sp}}
\newcommand{\id}{\mathbbm{1}}

\usetheme{default}
\usecolortheme{seahorse}

\title{Transizione di deconfinamento in 3D Yang-Mills con gruppo di gauge $\Sp(2)$ e studio degli effetti di stringa}

\author{\texorpdfstring{Studente: \textbf{Pini Nicholas}\\ \small Relatore: prof. \textbf{Giusti Leonardo}\\ \small Correlatore: prof. \textbf{Pepe Michele}}{Pini Nicholas}}

\institute
{
  Facoltà di Fisica Magistrale\\
  Università degli Studi di Milano Bicocca
}

\date
{Anno Accademico 2021/2022}

\titlegraphic{\hspace*{9cm}\includegraphics[width=2cm]{logo.png}}

\begin{document}
\beamertemplatenavigationsymbolsempty

\frame{\titlepage}


\begin{frame}
	\frametitle{Introduzione}

	\begin{itemize}
		\item Teorie di Yang-Mills presentano \alert{confinamento di colore}
		\item A temperature finite $\rightarrow$ transizione di deconfinamento
		\item Congettura di Svetisky e Yaffe
		\item Effective String Theory: modello efficace che descrive il potenziale 
	\end{itemize}

	\alert{Obiettivo}: studio della transizione di deconfinamento con gruppo di gauge $\Sp(2)$ con simulazioni su reticolo.

\end{frame}

\begin{frame}
	\frametitle{Teorie di gauge non abeliane}

	Azione di Yang-Mills di pura gauge nell'Euclideo:

	\begin{equation*}
		S_E = \frac{1}{2g^2} \int \dd[D]{x} \Tr[F_{\mu\nu}F_{\mu\nu}]
	\end{equation*}

	Path integral:

	\begin{equation*}
		\Z = \int \D{A} e^{-S_E[A]} \implies \expval{\mathcal{O}} = \frac{1}{\Z} \int \D{A} e^{-S_E[A]} \mathcal{O}
	\end{equation*}

	$\Z \rightarrow$ \alert{funzione di partizione} di un sistema statistico con fattore di Boltzmann $e^{-S_E[A]}$

\end{frame}

\begin{frame}
	\frametitle{Lattice gauge theory}

	Spaziotempo su reticolo con \alert{passo reticolare $a$} $\rightarrow$ cutoff al momento $p$.

	\alert{Azione di Wilson}:

	\begin{equation*}
		S_W = \frac{\beta}{N} \sum_x \sum_{\mu<\nu} \Re \Tr(\id - U_{\mu\nu}(x)) \qc \beta = \frac{2N}{g^2}
	\end{equation*}

	\begin{columns}
		\column{0.5\textwidth}
			\begin{itemize}
				\item per $a \rightarrow 0$, $S_W \rightarrow S_E$
				\item \alert{gauge invariante} per $a > 0$
				\item \alert{libertà asintotica}: $a \rightarrow 0 \implies g(a) \rightarrow 0$
			\end{itemize}

		\column{0.5\textwidth}
			Placchetta $U_{\mu\nu}$:

			\begin{figure}[h]
				\centering
				\includegraphics[width=0.7\textwidth]{plaquette.png}
				\caption{Gattringer, Christof, Lang, \textit{Quantum chromodynamics on the lattice}}
			\end{figure}
			
	\end{columns}
\end{frame}

\begin{frame}
	\frametitle{Potenziale d'interazione}
	\framesubtitle{Temperatura finita e Polyakov loop}

	Condizioni periodiche sul tempo $\rightarrow$ temperatura $T = \flatfrac{1}{(aN_t)}$

	\begin{columns}
		\column{0.5\textwidth}
			Loop di Polyakov:
			\begin{equation*}
					\phi(\va{x}) = \Tr[\prod_{j = 0}^{N_t - 1} U_0(j, \va{x})]
			\end{equation*}
			\begin{equation*}
				\begin{aligned}
					\expval{\phi(\va{x}) \phi(\va{y})^\dagger} &\sim e^{-\frac{1}{T} V(R, T)} \\
					&\sim e^{-\frac{1}{T}\sigma(T) R}						
				\end{aligned}
			\end{equation*}

		\column{0.5\textwidth}
			\begin{figure}[h]
				\centering
				\includegraphics[width=\textwidth]{polyloop.png}
				\caption{Suganuma et al., \textit{Interplay between Deconfinement and Chiral Properties}}
			\end{figure}
	\end{columns}

	\begin{columns}
		\column{0.5\textwidth}
			$\sigma(T) \rightarrow$ tensione di stringa a temperatura finita

		\column{0.5\textwidth}
			\begin{itemize}
				\item $\expval{\phi} = 0 \implies F_q \rightarrow \infty \rightarrow$ \alert{fase confinata} ($T < T_c$)
				\item $\expval{\phi} \neq 0 \implies F_q$ finita $\rightarrow$ \alert{fase deconfinata} ($T > T_c$)
			\end{itemize}
	\end{columns}

\end{frame}

\begin{frame}
	\frametitle{Congettura di Svetisky e Yaffe}
	\framesubtitle{Simmetria di centro}

	\begin{columns}
		\column{0.4\textwidth}
			$z \in$ centro gruppo gauge. Simmetria di centro:
			\begin{equation*}
				\phi(\va{x}) \rightarrow z \phi(\va{x})
			\end{equation*}

		\column{0.6\textwidth}
			\begin{figure}[h]
				\centering
				\includegraphics[width=0.6\textwidth]{polyloop_z.png}
			\end{figure}

	\end{columns}

	\

	\begin{itemize}
		\item $\expval{\phi} = 0 \implies$ simmetria mantenuta
		\item $\expval{\phi} \ne 0 \implies$ simmetria rotta spontaneamente
	\end{itemize}

	\

	\alert{Congettura di Svetisky e Yaffe}: se transizione di deconfinamento è di 2° ordine

	\

	\begin{columns}
		\column{0.35\textwidth}
			\centering
			Teoria di gauge $(d+1)$ dimensionale 

		\column{0.3\textwidth}
			\centering
			$\underleftrightarrow{\text{ classe di universalità }}$

		\column{0.35\textwidth}
			\centering
			Modello spin $d$ dimensionale

	\end{columns}

	\

	\begin{itemize}
		\item correlatore di Polyakov loop $\iff$ correlatore fra spin 
		\item fase deconfinata, $T > T_c$ $\iff$ fase ordinata, $T^{\text{spin}} < T_c^{\text{spin}}$
	\end{itemize}


\end{frame}

\begin{frame}
	\frametitle{Effective string theory}

	\begin{columns}
		\column{0.5\textwidth}
			EST: tubo di flusso $\rightarrow$ stringa vibrante
			\begin{itemize}
				\item modello effettivo a lunghe distanze
				\item termine di L{\"u}scher in $V(R)$
				\item estremamente predittivo
			\end{itemize}

		\column{0.5\textwidth}
			\begin{figure}[h]
				\centering
				\includegraphics[width=\textwidth]{string.png}
			\end{figure}			

	\end{columns}

	\

	\

	Per $R > \xi$, \alert{EST e congettura sono in accordo}:

	\begin{equation*}
		\expval{\phi(0) \phi(R)} \sim K_0(E_0 R) \quad \text{in 3D}
	\end{equation*}

\end{frame}

\begin{frame}
	\frametitle{EST vs modello di Ising}

	\begin{columns}
		\column{0.6\textwidth}
			Trans. 2° ordine $\implies \xi \sim \qty(1 - \frac{T}{T_c})^{-\nu}$

		\column{0.4\textwidth}
			\alert{Vicino al punto critico}: $\xi \rightarrow \infty \implies R < \xi$
	\end{columns}

	\

	In 3 dimensioni:

	\
	
	\begin{columns}
		\column{0.5\textwidth}
			\centering
			Congettura prevede $\nu = 1$ (modello di Ising)

		\column{0.5\textwidth}
			\centering
			EST prevede $\nu = \flatfrac{1}{2}$

	\end{columns}
	
	\

	\
	
	A $T \lesssim T_c$, EST non è più predittiva $\rightarrow$ \alert{stringa dissolta da fluttuazioni e potenziale schermato}.

	\

	\begin{columns}
		\column{0.7\textwidth}
			Per $R < \xi$, congettura prevede
			\small
			\begin{equation*} 
				\begin{aligned}
					\expval{\phi(0) \phi(R)} &= \frac{k_s}{R^{\flatfrac{1}{4}}} \biggl[ 1 + \frac{R}{2\xi} \ln(\frac{e^{\gamma_e}R}{8\xi}) + \frac{R^2}{16\xi^2} \\
					&+ \frac{R^3}{32\xi^3} \ln(\frac{e^{\gamma_e}R}{8\xi}) + O\qty(\frac{R^4}{\xi^4}\ln^2\frac{R}{\xi}) \biggr]
				\end{aligned}
			\end{equation*}

		\column{0.3\textwidth}
			\fbox{
				\begin{minipage}{\textwidth}
					\alert{Scopo}: 
					
					studiare $\expval{\phi(0) \phi(R)}$ per $T \lesssim T_c$ con gruppo di gauge $\Sp(2)$
				\end{minipage}
			}

	\end{columns}
\end{frame}

\begin{frame}
	\frametitle{Risultati}
	\framesubtitle{Misura loop di Polyakov}

	Reticolo 3D con \alert{$N_s = 40,60,80,100$} e \alert{$N_t = 5,6,7,8$}.

	\begin{columns}
		\column{0.4\textwidth}
			\begin{figure}[h]
				\centering
				\vspace{-1em}
				\includegraphics[width=1.1\textwidth]{mchistory_nt=6_l=100_beta=27.png}
				\includegraphics[width=1.1\textwidth]{mchistory_nt=6_l=100_beta=28.png}
			\end{figure}
			
		\column{0.6\textwidth}
			\begin{figure}[h]
				\includegraphics[width=\textwidth]{polydist_nt=6_l=100.png}
			\end{figure}

			Comportamento tipico di transizione di fase di secondo ordine:
			\begin{itemize}
				\item fase confinata: $\expval{|\phi|} = 0$
				\item fase deconfinata: $\expval{|\phi|} \ne 0$ e \alert{eventi di tunneling}
			\end{itemize}
			
	\end{columns}
	

\end{frame}

\begin{frame}
	\frametitle{Risultati}
	\framesubtitle{Misura temperatura critica}

	$T = \flatfrac{1}{(a N_t)} \implies$ cerchiamo $\beta_c(N_t)$ tale che il sistema è nel punto critico.

	\

	\alert{Suscettività}: misura la larghezza della distribuzione del loop di Polyakov.

	\begin{columns}
		\column{0.45\textwidth}
			\begin{equation*}
				\chi = \sum_{\va{x}} \expval{\phi(\va{0}) \phi(\va{x})} = N_s^2 \expval{\phi^2}
			\end{equation*}
		
			\

			\begin{equation*}
				\begin{aligned}
					\chi(\beta) &\sim a + b\qty(\beta^{(0)} - \beta)^2 \\
					&+ c\qty(\beta^{(0)} - \beta)^3 + d\qty(\beta^{(0)} - \beta)^4					
				\end{aligned}
			\end{equation*}


		\column{0.55\textwidth}
		\begin{figure}[h]
				\includegraphics[width=\textwidth]{susc_all_l/suscplot_nt=6.png}
				\caption{$N_t = 6$}
			\end{figure}
						
	\end{columns}
\end{frame}

\begin{frame}
	\frametitle{Risultati}
	\framesubtitle{$\beta$ critici}

	\begin{columns}
		\column{0.5\textwidth}
			\scriptsize
			\centering
			\begin{tabular}{|c|c|c|c|}
				\hline
				$N_t$ & $N_s$ & $\beta_c$ & $\chi^2$ \\
				\hline
				\multirow{4}{*}{5} 
				& 40  & $23.312(14)$   & 1.0933\\
				& 60  & $23.2748(52)$ & 1.2741 \\
				& 80  & $23.2886(64)$ & 0.5137 \\
				& 100 & $23.2817(46)$ & 1.4615 \\
				\hline
				\multirow{4}{*}{6} 
				& 40  & $27.589(30)$  & 0.781 \\
				& 60  & $27.547(10)$  & 0.7684 \\
				& 80  & $27.537(12)$ & 0.559 \\
				& 100 & $27.566(13)$ & 1.6649 \\
				\hline
				\multirow{4}{*}{7} 
				& 40  & $32.103(31) $  & 0.1255 \\
				& 60  & $31.8149(92)$ & 0.7616 \\
				& 80  & $31.8190(97)$  & 0.6526 \\
				& 100 & $31.8299(99)$ & 1.328 \\
				\hline
				\multirow{4}{*}{8} 
				& 40  & $36.275(68)$ & 0.7747 \\
				& 60  & $36.103(22)$ & 0.8324 \\
				& 80  & $36.065(19)$ & 1.3907 \\
				& 100 & $36.065(14)$ & 1.0921 \\
				\hline
			\end{tabular}

			\

			\normalsize
			Per $N_s = 100$, \alert{effetti di volume finito sono piccoli} $\rightarrow$ valori di $\beta_c$ 
			validi nel limite termodinamico.

		\column{0.5\textwidth}
			\centering
			\begin{figure}[h]
				\centering
				\includegraphics[width=0.7\textwidth]{beta_vs_l/beta_vs_l_nt=5.png}

				\includegraphics[width=0.7\textwidth]{beta_vs_l/beta_vs_l_nt=6.png}

				\includegraphics[width=0.7\textwidth]{beta_vs_l/beta_vs_l_nt=7.png}

				\includegraphics[width=0.7\textwidth]{beta_vs_l/beta_vs_l_nt=8.png}
			\end{figure}
			
	\end{columns}

\end{frame}

\begin{frame}
	\frametitle{Risultati}
	\framesubtitle{Finite size scaling}
	
	\alert{Finite size scaling}:
	
	\begin{itemize}
		\item osservabili riscalate a volume finito descrivono la stessa curva 
		\item teoria di gauge 3D nella classe di universalità di Ising 
	\end{itemize}
	
	\begin{columns}
		\column{0.5\textwidth}
			\begin{figure}[h]
				\centering
				\includegraphics[width=\textwidth]{fss/modphi_fss_nt=6.png}
				\includegraphics[width=\textwidth]{fss/phi2_fss_nt=6.png}
				\caption{$N_t = 6$. $\beta_c$ non aggiustati.}
			\end{figure}
		\column{0.5\textwidth}
			\begin{figure}[h]
				\centering
				\includegraphics[width=\textwidth]{adj_fss/adj_modphi_fss_nt=6.png}
				\includegraphics[width=\textwidth]{adj_fss/adj_phi2_fss_nt=6.png}
				\caption{$N_t = 6$. $\beta_c$ aggiustati.}
			\end{figure}
	\end{columns}
\end{frame}

\begin{frame}
	\frametitle{Risultati}
	\framesubtitle{$\beta_c$ in funzione di $N_t$}

	\begin{columns}
		\column{0.5\textwidth}
			Fissato $N_s = 100$, fittiamo $\beta_c$ al variare di $N_t$ usando una retta. 
			
			
		\column{0.5\textwidth}
		\begin{equation*}
			\begin{gathered}
				\beta_c(N_t) \sim a + bN_t \\
				\chi^2 = 2.5612
			\end{gathered}
		\end{equation*}
	\end{columns}

	\begin{figure}[h]
		\centering
		\includegraphics[width=\textwidth]{linear_betafit.png}
	\end{figure}
\end{frame}

\begin{frame}
	\frametitle{Risultati}
	\framesubtitle{Lunghezza di correlazione}

	\begin{columns}
		\column{0.66\textwidth}
			\scriptsize
			\centering
			\begin{tabular}{|c|c|c|c|c|c|}
				\hline
				$N_t$ & $N_s$ & $\flatfrac{T}{T_c} $ & $R$ & $\xi$ & $\chi^2$ \\
				\hline
				\multirow{2}{*}{5} & \multirow{2}{*}{100} & \multirow{2}{*}{0.95}
						& $(2, 13)$ & $14.157(66)$ & 1.433 \\
					& & & $(12, 50)$ & $14.26(58)$ & 1.4986 \\
				\hline
				\multirow{2}{*}{6} & \multirow{2}{*}{100} & \multirow{2}{*}{0.95}
						& $(2, 17)$ & $17.42(11)$ & 0.8582 \\
					& & & $(14, 42)$ & $16.86(12)$ & 0.513 \\
				\hline
				\multirow{2}{*}{7} & \multirow{2}{*}{100} & \multirow{2}{*}{0.95}
						& $(2, 20)$ & $20.97(15)$ & 0.5012 \\
					& & & $(17, 50)$ & $20.66(17)$ & 0.4379 \\
				\hline
				\multirow{2}{*}{8} & \multirow{2}{*}{100} & \multirow{2}{*}{0.95}
						& $(1, 24)$ & $24.95(22)$ & 2.9113 \\
					& & & $(18, 50)$ & $24.26(28)$ & 0.4311 \\
				\hline
			\end{tabular}

		\column{0.34\textwidth}
			Misura complicata dalla scelta limita di valori di $R$ se $\xi$ è troppo grande o troppo piccolo.			

	\end{columns}

	\vspace{\fill}

	\begin{adjustwidth}{-1cm}{-1cm}
		\begin{figure}[b]
			\centering
			\includegraphics[width=0.55\textwidth]{corr_t095/corr_short_fit_nt6.png}
			\includegraphics[width=0.55\textwidth]{corr_t095/corr_long_fit_nt6.png}
			\caption{$N_t = 6$}
		\end{figure}
	\end{adjustwidth}

\end{frame}

\begin{frame}
	\frametitle{Conclusioni}
	\framesubtitle{Risultati}

	Sia per $R < \xi$ che per $R > \xi$, misure compatibili di $\xi$ $\implies$

	$\implies$ \alert{congettura di Svetisky e Yaffe prevede correttamente andamento di $\expval{\phi(0) \phi(R)}$} nelle vicinanze
	del punto critico.

	\

	Inoltre, come ci aspettavamo:

	\begin{itemize}
		\item teoria di Yang-Mills 3D con gruppo di gauge $\Sp(2)$ presenta una transizione di deconfinamento del
		secondo ordine
		\item andamento di $\expval{\phi(0)\phi(R)}$ per $R > \xi$ previsto da EST
		\item EST a $R < \xi$ non è più valida
	\end{itemize}

\end{frame}

\begin{frame}
	\frametitle{Conclusioni}
	\framesubtitle{Ricerche future}
	
	Possibili ricerche future:

	\

	\begin{itemize}
		\item studiare $\sigma(T)$ in $V(R, T)$
		\item studiare come si passa da descrizione EST a descrizione della congettura
	\end{itemize}

\end{frame}

\begin{frame}

	\centering
	\Huge
	Grazie per l'attenzione!	

\end{frame}

\end{document}