\documentclass[reqno,12pt]{article}

\usepackage[utf8]{inputenc}
\usepackage[tbtags]{amsmath}
\usepackage[english]{babel}
\usepackage{bm}
\usepackage{bbm} % \mathbbm{1}
\usepackage{amsfonts}
\usepackage{microtype}
\usepackage{physics}
\usepackage{tensor}
\usepackage{slashed}
\usepackage{subfig}
\usepackage{mathtools}
\usepackage[subfigure]{tocloft}
\usepackage[parfill]{parskip}
\usepackage{multirow}
\usepackage{xcolor}
\usepackage{biblatex}
\usepackage{csquotes} % required by biblatex
\addbibresource{source.bib}

\numberwithin{equation}{section}

\renewcommand{\cftsecleader}{\cftdotfill{\cftdotsep}}

\newcommand{\red}[1]{\textbf{\textcolor{red}{#1}}}

\newcommand{\SU}{\mathrm{SU}}
\newcommand{\Sp}{\mathrm{Sp}}
\newcommand{\id}{\mathbbm{1}}

\title{}
\date{}
\author{}

\begin{document}

\begin{abstract}
	abstract
\end{abstract}

\section{Introduction}
\red{Brief intro about the paper. Past work, theory, references.}


\section{Theoretical background}
\red{Explain the concepts of lattice gauge theory, Wilson and Polyakov loops, center symmetry and Sp(2) group. 
Mostly from Pepe's work. Read the paper by Caselle to see if there is something to pure here (SY conjecture? String tension? 
Universality with spin model? EST?)}

\section{Simulation and algorithm}
Define the partition function of the system to be

\begin{equation}
	\mathcal{Z} = \int \mathcal{D}U e^{-\beta S(U)} \qq{where} \int \mathcal{D} U \equiv \prod_{x,\mu} \int \dd{U_{x,\mu}}.
\end{equation}

$U_{x,\mu}$ is an element of the group (a link variable starting at the lattice site $x$, pointing
along the direction $\mu$), 
and $S$ is the action, defined as
\begin{equation}
	S = \sum_\square S_\square \qc 
	S_\square \equiv - \frac{1}{4} \Tr(U_{x,\mu}U_{x+\hat{\mu}, \nu}U^\dagger_{x+\hat{\nu}, \mu}U^\dagger_{x,\nu}).
\end{equation}
The symbol $\square$ represents the plaquette: the minimum loop possible on the lattice. Note that a link 
variable has a direction: the adjoint of a link variable is the link connecting the two sites in the opposite
direction. In other words:
\begin{equation}
	U_{x,\mu} = U_{x+\hat{\mu}, -\mu}^\dagger.
\end{equation}

Note that the lattice has periodic boundary conditions.

The algorithm basic idea is to generate a new link element $U$ with a Boltzmann probability distribution:
\begin{equation}
	P(U) = \frac{1}{\mathcal{Z}} e^{-\beta S(U)} \dd{U}.
\end{equation}  

While this is fairly easy for the SU(2) gauge group using Creutz's algorithm \cite{creutz}, there is no obvious
way to generalize it to other gauge groups, like SU(N) or Sp(N). Thus, to generate new links belonging to Sp(2), we use
a more general approach, as designed by Cabibbo and Marinari \cite{cabibbo}.  

We consider a set $F$ of SU(2) subgroups of the gauge group Sp(2). Given a Sp(2) element $U$ of the form
\begin{equation}
	U = \mqty(
		W_{11} & W_{12} & X_{11} & X_{12} \\
		W_{21} & W_{22} & X_{21} & X_{22} \\
		X^*_{22} & -X^*_{21} & W^*_{22} & -W^*_{21} \\
		-X^*_{12} & X^*_{11} & -W^*_{12} & W^*_{11}
	),
\end{equation}
where $W_{ij}, X_{kl} \in \mathbb{C}$ for $i,j,k,l = 1, 2$, 
we can construct four SU(2) subgroups, by extracting two complex numbers $t_1$ and $t_2$:
\begin{itemize}
	\item $\begin{cases} t_1 = W_{11} \\ t_2 = X_{12} \end{cases}$
	\item $\begin{cases} t_1 = W_{22} \\ t_2 = X_{21} \end{cases}$
	\item $\begin{cases} t_1 = W_{11} + W_{22} \\ t_2 = X_{11} - X_{22} \end{cases}$
	\item $\begin{cases} t_1 = W_{11} + W^*_{22} \\ t_2 = W_{12} - W^*_{21} \end{cases}$
\end{itemize}
We can then build an SU(2) group element as
\begin{equation}
	a_k = \mqty(t_1 & t_2 \\ -t_2^* & t_1^*) \qc k = 1, 2, 3, 4,
\end{equation}
where $k$ labels each subgroup. 
Each choice of $t_1$ and $t_2$ above gives a different SU(2) element belonging to a SU(2) subgroup of Sp(2).

We define $A_k$ to be an SU(2) element belonging to the $k$th SU(2) subgroup, embedded into Sp(2). For each subgroup in $F$,
the Sp(2) embedding is constructed as follows:
\begin{itemize}
	\item $A_1 = \mqty(
		t_1 & 0 & 0 & t_2 \\
		0 & 1 & 0 & 0 \\
		0 & 0 & 1 & 0 \\
		-t^*_2 & 0 & 0 & t^*_1 
		)$
	\item $A_2 = \mqty(
		1 & 0 & 0 & 0 \\
		0 & t_1 & t_2 & 0 \\
		0 & -t^*_2 & t^*_1 & 0 \\
		0 & 0 & 0 & 1 
		)$
	\item $A_3 = \mqty(
		t_1 & 0 & t_2 & 0 \\
		0 & t_1 & 0 & -t_2 \\
		-t_2^* & 0 & t^*_1 & 0 \\
		0 & t^*_2 & 0 & t^*_1 
		)$
	\item $A_4 = \mqty(
		t_1 & t_2 & 0 & 0 \\
		-t^*_2 & 1 & 0 & 0 \\
		0 & 0 & t_1 & t_2  \\
		0 & 0 & -t^*_2 & 1 
		)$
\end{itemize}

Generating each $A_k$ randomly, we define the new link $U'$ to be

\begin{equation}
	U' = \qty(\prod_k A_k) U.
\end{equation}

It is proven \cite{cabibbo} that this algorithm leads to thermalization, if each $a_k$ is randomly distributed as

\begin{equation} \label{eq:probdist}
	P(A_k) = \dd{A_k} \frac{e^{-\beta S(A_k U_{k-1})}}{\mathcal{Z}_k(U_{k-1})},
\end{equation}

where $U_{k-1} = \qty(\prod_1^{k-1}A_k) U$ with $U_0 = U$ and 

\begin{equation}
	\mathcal{Z}_k(U) = \int_{SU(2)_k} \dd{A} e^{-\beta S(AU)}.
\end{equation}


The reason for the decomposition into SU(2) subgroups is to efficiently generate $A_k$ according to 
\eqref{eq:probdist}. In fact, now that we are dealing with SU(2) elements, we can fall back to Creutz's
algorithm \cite{creutz} to generate each SU(2) element, embed it into Sp(2) as explained above, and left multiply
the original link $U$ by it. 

Focusing on a single link $U$ to update, we are interested only in the plaquettes that contain $U$. Defining
$\widetilde{U}$ to be the product of the staples surrounding $U$ (an ordered product of the three links in the plaquettes that
are not $U$ itself), we have 

\begin{align}
	S(A_k U) &= -\frac{1}{4} \Tr(A_k U \sum_i \widetilde{U}_i) + \text{terms independent of } A_k \\
	&= -\frac{1}{4} \Tr(a_k u_k \sum_i \tilde{u}_k^i) + \text{terms independent of } a_k, 
\end{align}

where $a_k$, $u_k$ and $\tilde{u}_k$ are SU(2) elements corresponding to the $k$th
subgroup extracted from $A_k$, $U$ and $\widetilde{U}$, respectively. This implies that we want to generate
$a_k$ according to the distribution

\begin{equation}
	\dd{P(a_k)} \sim e^{\frac{1}{4}\beta\Tr(a_k u_k \sum_i \tilde{u}_k^i)} \dd{a_k}.
\end{equation}

We parametrize $a_k$ as

\begin{equation}
	a_k = \alpha_0 \id + i \vec{\alpha} \cdot \vec{\sigma},
\end{equation}

where $\alpha_\mu \in \mathbb{C} \ \forall \mu = 1, 2, 3, 4$ with the constraint that 

\begin{equation}
	\alpha^2 \equiv \alpha_0^2 + |\vec{\alpha}|^2 = 1
\end{equation} 

and $\vec{\sigma} = \qty(\sigma_1, \sigma_2, \sigma_3)$ is the three-vector of the $2 \times 2$ Pauli matrices. 

The SU(2) group measure is 
\begin{equation}
	\dd{a_k} = \frac{1}{2\pi^2} \delta(\alpha^2 - 1) \dd[4]{\alpha}.
\end{equation}

Since the sum of SU(2) elements is proportional to an SU(2) element, we write
\begin{equation}
	u_k \sum_i \tilde{u}_k^i = c \bar{u}_k \qc \bar{u}_k \in \SU(2)
\end{equation}
where 
\begin{equation}
	c = \det(u_k \sum_i \tilde{u}_k^i)^{\flatfrac{1}{2}}.
\end{equation}

The probability distribution for $a_k$ now becomes
\begin{equation}
	\dd{P(a_k)} \sim e^{\frac{1}{4}\beta\Tr(c a_k \bar{u}_k)} \dd{a_k}.
\end{equation}

The group measure is invariant under multiplication by another SU(2) element:
\begin{equation}
	\dd(b a_k) = \dd{a_k} \qfor b \in \SU(2),
\end{equation}

so that we can write
\begin{equation}
	\dd{P(a_k \bar{u}_k^{-1})} \sim e^{\frac{1}{4}\beta c \Tr(a_k)} \dd{a_k} 
	= \frac{1}{2\pi^2} e^{\frac{\beta}{2}c \alpha_0} \delta(\alpha^2 - 1) \dd[4]{\alpha},
\end{equation}

because $\Tr(a_k) = 2\alpha_0$. Noting that $\delta(\alpha^2 - 1)\dd[4]{\alpha} = 
\frac{1}{2}(1 - \alpha_0^2)^{\flatfrac{1}{2}} \dd{\alpha_0}\dd{\Omega}$, we rewrite $\dd{P(a_k \bar{u}_k^{-1})}$ as
\begin{equation}
	\dd{P(a_k \bar{u}_k^{-1})} \sim 
	\frac{1}{2\pi^2} \frac{1}{2}(1 - \alpha_0^2)^{\flatfrac{1}{2}} e^{\frac{\beta}{2}c \alpha_0} \dd{\alpha_0}\dd{\Omega}
\end{equation}
with $\alpha_0 \in (-1,1)$ and $\vec{\alpha}$ is a totally random unit three-vector.

The problem is now generating the four-vector $\alpha_\mu$ according to the distribution above, thus obtaining
$a_k \in \SU(2)$. Finally, we obtain $A_k$ by embedding $a_k \bar{u}_k^{-1} \in \SU(2)$ into Sp(2). Doing this for every 
SU(2) subgroup will yield the new link $U'$.

To generate $a_k$, we have to randomly generate $\alpha_0$ according to

\begin{equation} \label{eq:probdistalpha}
	P(\alpha_0) \sim (1 - \alpha_0^2)^{\flatfrac{1}{2}} e^{\frac{\beta}{2}c\alpha_0}.
\end{equation}

The algorithm is quite simple. We uniformly generate $x$ in the range 

\begin{equation}
	e^{-\beta c} < x < 1
\end{equation} 

and define a trial $\alpha_0$ distributed according to $e^{\frac{\beta}{2}c\alpha_0}$ as

\begin{equation}
	\alpha_0 = 1 + \frac{2}{\beta c}\ln{x}.
\end{equation}

To account for the term $\qty(1 - \alpha_0^2)^{\flatfrac{1}{2}}$ in \eqref{eq:probdistalpha}, we \textit{reject} this trial
$\alpha_0$ with probability $1 - \qty(1 - \alpha_0^2)^{\flatfrac{1}{2}}$, generating a new trial $\alpha_0$ if 
the rejection is successful. We keep doing this until a trial $\alpha_0$ is finally accepted. 

The unit vector $\vec{\alpha} = (\alpha_1, \alpha_2, \alpha_3)$ is constructed by uniformly generating 

\begin{gather}
	0 < \phi < 2\pi, \\
	-1 < y < 1
\end{gather}

and defining 

\begin{align}
	\theta &\equiv \arccos(y), \\
	r &\equiv \qty(1 - \alpha_0)^{\flatfrac{1}{2}}.
\end{align}

We finally have

\begin{equation}
	\begin{cases}
		\alpha_1 = r \sin(\theta) \cos(\phi) \\
		\alpha_2 = r \sin(\theta) \sin(\phi) \\
		\alpha_3 = r \cos(\theta)
	\end{cases}.
\end{equation}


\section{Results}
\red{Put plots and results of the fit and simulations: fit of susceptibility peaks, fit of beta vs nt, etc.}

\printbibliography
% https://www.overleaf.com/learn/latex/Bibliography_management_in_LaTeX

\end{document}