\documentclass[reqno,12pt]{article}

\usepackage[utf8]{inputenc}
\usepackage[tbtags]{amsmath}
\usepackage[english]{babel}
\usepackage{bm}
\usepackage{amsfonts}
\usepackage{microtype}
\usepackage{physics}
\usepackage{tensor}
\usepackage{slashed}
\usepackage{subfig}
\usepackage{mathtools}
\usepackage[subfigure]{tocloft}
\usepackage[parfill]{parskip}
\usepackage{multirow}
\usepackage{biblatex}
\usepackage{csquotes} % required by biblatex
\addbibresource{source.bib}

\numberwithin{equation}{section}

\renewcommand{\cftsecleader}{\cftdotfill{\cftdotsep}}

\newcommand{\SU}{\mathrm{SU}}
\newcommand{\Sp}{\mathrm{Sp}}

\title{}
\date{}
\author{}

\begin{document}

\begin{abstract}
	abstract
\end{abstract}

\section{Introduction}
Brief intro about the paper. Past work, theory, references.


\section{Theoretical background}
Explain the concepts of lattice gauge theory, Wilson and Polyakov loops, center symmetry and Sp(2) group. 
Mostly from Pepe's work.

Read the paper by Caselle to see if there is something to pure here (SY conjecture? String tension? 
Universality with spin model? EST?)

\section{Simulation and algorithm}
\begin{center}
	\textit{Introduce the algorithm used in the simulation. Creutz, Cabibbo Marinari. Explain in details what the code
	does (without explicitly pasting the code): for example, how to extract SU(2) matrices from Sp(2).}
\end{center}

Define the partition function of the system to be

\begin{equation}
	\mathcal{Z} = \int \mathcal{D}U e^{-\beta S(U)} \qq{where} \int \mathcal{D} U \equiv \prod_{x,\mu} \int \dd{U_{x,\mu}}.
\end{equation}

$U_{x,\mu}$ is an element of the group (a link variable starting at the lattice site $x$, pointing
along the direction $\mu$), 
and $S$ is the action, defined as
\begin{equation}
	S = \sum_\square S_\square \qc 
	S_\square \equiv 1 - \frac{1}{4} \Tr(U_{x,\mu}U_{x+\hat{\mu}, \nu}U^\dagger_{x+\hat{\nu}, \mu}U^\dagger_{x,\nu}).
\end{equation}
The symbol $\square$ represents the plaquette: the minimum loop possible on the lattice. Note that a link 
variable has a direction: the adjoint of a link variable is the link connecting the two sites in the opposite
direction. In other words:
\begin{equation}
	U_{x,\mu} = U_{x+\hat{\mu}, -\mu}^\dagger.
\end{equation}

The idea is to follow the \textit{heat-bath} algorithm as designed by Cabibbo and Marinari \cite{cabibbo}, and generate a new link
element with a Boltzmann probability distribution:
\begin{equation}
	P(U) = \frac{1}{\mathcal{Z}} e^{-\beta S(U)} \dd{U}.
\end{equation}  

To do this, we consider a set $F$ of SU(2) subgroups of the gauge group Sp(2). Given a Sp(2) element $U$ of the form
\begin{equation}
	U_\text{Sp(2)} = \mqty(
		W_{11} & W_{12} & X_{11} & X_{12} \\
		W_{21} & W_{22} & X_{21} & X_{22} \\
		X^*_{22} & -X^*_{21} & W^*_{22} & -W^*_{21} \\
		-X^*_{12} & X^*_{11} & -W^*_{12} & W^*_{11}
	),
\end{equation}
where $W_{ij}, X_{kl} \in \mathbb{C}$ for $i,j,k,l = 1, 2$, 
we can construct four SU(2) subgroups, by extracting two complex numbers $t_1$ and $t_2$ as
\begin{itemize}
	\item $\begin{cases} t_1 = W_{11} \\ t_2 = X_{12} \end{cases}$
	\item $\begin{cases} t_1 = W_{22} \\ t_2 = X_{21} \end{cases}$
	\item $\begin{cases} t_1 = W_{11} + W_{22} \\ t_2 = X_{11} - X_{22} \end{cases}$
	\item $\begin{cases} t_1 = W_{11} + W^*_{22} \\ t_2 = W_{12} - W^*_{21} \end{cases}$
\end{itemize}
and building an SU(2) group element as
\begin{equation}
	U_\text{SU(2)} = \mqty(t_1 & t_2 \\ -t_2^* & t_1^*).
\end{equation}
Each choice of $t_1$ and $t_2$ above gives a different SU(2) element belonging to a SU(2) subgroup of Sp(2).

\section{Results}
Put plots and results of the fit and simulations: fit of susceptibility peaks, fit of beta vs nt, etc.


To understand how EST works, let us describe 

\printbibliography
% https://www.overleaf.com/learn/latex/Bibliography_management_in_LaTeX

\end{document}