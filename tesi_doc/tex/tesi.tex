\documentclass[reqno, 12pt]{article}

\usepackage[utf8]{inputenc}
\usepackage[tbtags]{amsmath}
\usepackage[english]{babel}
\usepackage{bm}
\usepackage{amsfonts}
\usepackage{microtype}
\usepackage{physics}
\usepackage{tensor}
\usepackage{slashed}
\usepackage{subfig}
\usepackage{mathtools}
\usepackage[subfigure]{tocloft}
\usepackage[parfill]{parskip}
\usepackage{multirow}
%\usepackage{biblatex}
%\usepackage{csquotes} % required by biblatex
%\addbibresource{source.bib}

\renewcommand{\cftsecleader}{\cftdotfill{\cftdotsep}}

\title{}
\date{}
\author{}

\begin{document}

\begin{abstract}
	abstract
\end{abstract}

\section{Introduction}
Brief intro about the paper. Past work, theory, references.


\section{Theoretical background}
Explain the concepts of lattice gauge theory, Wilson and Polyakov loops, center symmetry and Sp(2) group. 
Mostly from Pepe's work.

Read the paper by Caselle to see if there is something to pure here (SY conjecture? String tension? 
Universality with spin model? EST?)

\section{Simulation and algorithm}
Introduce the algorithm used in the simulation. Creutz, Cabibbo Marinari. Explain in details what the code
does (without explicitely pasting the code): for example, how to extract SU(2) matrices from Sp(2).

\section{Results}
Put plots and results of the fit and simulations: fit of susceptibility peaks, fit of beta vs nt, etc.


To understand how EST works, let us describe 

%\printbibliography
% https://www.overleaf.com/learn/latex/Bibliography_management_in_LaTeX

\end{document}