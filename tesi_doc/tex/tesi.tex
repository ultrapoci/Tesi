\documentclass[reqno,12pt]{article}

\usepackage[utf8]{inputenc}
\usepackage[tbtags]{amsmath}
\usepackage[english]{babel}
\usepackage{bm}
\usepackage{bbm} % \mathbbm{1}
\usepackage{amsfonts}
\usepackage{microtype}
\usepackage{physics}
\usepackage{tensor}
\usepackage{slashed}
\usepackage{subfig}
\usepackage{mathtools}
\usepackage[subfigure]{tocloft}
\usepackage[parfill]{parskip}
\usepackage{multirow}
\usepackage{xcolor}
\usepackage{biblatex}
\usepackage{csquotes} % required by biblatex
\addbibresource{source.bib}

\numberwithin{equation}{section}

\renewcommand{\cftsecleader}{\cftdotfill{\cftdotsep}}

\newcommand{\red}[1]{\textbf{\textcolor{red}{#1}}}

\newcommand{\SU}{\mathrm{SU}}
\newcommand{\Sp}{\mathrm{Sp}}
\newcommand{\id}{\mathbbm{1}}

\title{Tesi}
\date{2022}
\author{Nicholas Pini}


\begin{document}

\maketitle

\begin{abstract}
	abstract
\end{abstract}

\tableofcontents

\section{Introduction}
\red{Brief intro about the paper. Past work, theory, references.}


\section{Theoretical background}
\red{Explain the concepts of lattice gauge theory, Wilson and Polyakov loops, center symmetry and Sp(2) group. 
Mostly from Pepe's work. Read the paper by Caselle to see if there is something to pure here (SY conjecture? String tension? 
Universality with spin model? EST?)}

\subsection{Lattice gauge theory}
In this work, we will be focusing on a (2+1) dimensional lattice gauge theory (two spatial dimensions and one time dimension).
We will present a brief summary of lattice gauge theories (LGT).

An LGT is a gauge theory built on a discretized spacetime: a lattice of spacing $a$ with $N_s$ sites in the space directions
and $N_t$ sites in the time direction. This discretization is useful, because it introduces a natural cutoff for the momentum $p$, thus 
regularizing the theory:

\begin{equation}
	p \in \qty(-\frac{\pi}{a}, \frac{\pi}{a}).
\end{equation}

Starting from the continuum limit action

\begin{equation}
	\widetilde{S} = \frac{1}{2g_0^2} \int \dd[4]{x} \Tr(F_{\mu\nu}F^{\mu\nu}),
\end{equation}

where $F_{\mu\nu} = F_{\mu\nu}^a T^a = \partial_\mu A_\nu - \partial_\nu A_\mu - [A_\mu, A_\nu]$, we define the \textit{link} variable
as

\begin{equation}
	U_\mu(x) = e^{-ia A_\mu(x)}.	
\end{equation}

Given a gauge transformation $G(x)$, a link variable transforms as

\begin{equation}
	U_\mu(x) \longrightarrow G(x) U_\mu(x) G^\dagger(x+\hat{\mu}).
\end{equation}

The link variable can be thought of as connecting the site at $x$ to the site at $x + \hat{\mu}$, and it belongs to the gauge
group defining the theory. We define a \textit{plaquette} $U_{\mu\nu}(x)$ as the smallest loop of product of link variables:

\begin{equation}
	U_{\mu\nu}(x) = U_\mu(x) U_\nu(x+\hat{\mu}) U^\dagger_\mu(x+\hat{\nu}) U^\dagger_\nu(x).
\end{equation}

Note that links are oriented, and the adjoint operation flips a link, inverting its direction.  

Finally, we can define the discretized action of an LGT:

\begin{equation} \label{eq:lgt_action}
	S = -\frac{\beta}{N} \sum_x \sum_{\mu < \nu} \Tr U_{\mu\nu}(x) = -\frac{2}{g_0^2} \sum_x \sum_{\mu < \nu} \Tr U_{\mu\nu}(x),
\end{equation}

where $g_0$ is the bare gauge coupling, $N$ is the trace normalization constant related to the gauge group, and
$\beta \equiv \frac{2N}{g_0^2}$. This action is invariant under gauge transformations: in fact, despite
links not being gauge invariant, a plaquette is indeed gauge invariant.

The path integral is

\begin{equation}
	\mathcal{Z} = \int \mathcal{D}U e^{-S}
\end{equation}

for 

\begin{equation}
	\int \mathcal{D}U = \prod_{x, \mu} \int_{\substack{\text{gauge} \\ \text{group}}} \dd{U_{x, \mu}}. 
\end{equation}

\subsection{Static quarks and Wilson loops}
We want to study the confinement-deconfinement phase transition that is typical of QCD. To do so, we imagine instantly
creating a static quark-antiquark pair at distance $R$ from one another. "Static" means that these quarks have infinite
mass and are thus non-dynamical. We let them evolve with time for a duration of $T$, 
and finally we instantly destroy them. In the context of a lattice gauge theory, the line connecting the two quarks has swept an area
of $R \times T$. This picture is described by \textit{Wilson loops}. 

A Wilson loop is the trace of an ordered product of link variables on an LGT along some path $\mathcal{C}$:

\begin{equation}
	W(\mathcal{C}) = \Tr \prod_{t, \vec{x}} U_\mu(t, \vec{x}).
\end{equation}

If the path $\mathcal{C}$ is a $R \times T$ rectangle, we are basically considering the rectangle swept by the quark pair: the Wilson
loop is then interpreted as the free energy. In fact, it is possible to relate the potential $V(R)$ 
between these two quarks to the expectation value of the Wilson loop for the $R \times T$ rectangle:

\begin{equation}
	V(R) = - \lim_{L\rightarrow\infty} \frac{1}{L} \log{\expval{W(\mathcal{C})}}.
\end{equation}

In a confining theory, the potential $V(R)$ is linearly rising, and the Wilson loop exhibits a so-called "area law":

\begin{equation}
	\expval{W(\mathcal{C})} \sim e^{-\sigma_0 R T}.
\end{equation}

$\sigma_0$ is called \textit{zero temperature string tension}. "String tensions" comes from the fact that the interquark 
potential can be effectively described as a vibrating string \cite{caselle}.


A complete introduction to Wilson loops and their relation to the interquark potential can be found in \cite{gattringer}. 

\subsection{Finite temperature LGT and Polyakov loops}
The description of the confining potential through Wilson loops does not take into account the temperature of the system.
If we want to study how the confinement-deconfinement transition depends on temperature, we need to define the theory at finite
temperature in the first place. 

It is well known how to do so for an LGT defined on a $(d+1)$ dimensional lattice of volume $(N_s a)^d(N_t a)$:
we consider the lattice regularization and impose \textit{periodic boundary condition} 
for bosonic fields (which are the fields we will be considering in this work). In particular, the compactified time direction
is interpreted as the temperature $T$ of the system, through the relation

\begin{equation} \label{eq:lattice_temperature}
	T = \frac{1}{N_t a}.
\end{equation}

Thus, the length of the lattice in the time direction is inversely proportional to the temperature of the system.

The Wilson loop is not suitable anymore in describing the interquark potential in a finite temperature context. Fortunately,
there is another quantity we can consider: the correlation function of two \textit{Polyakov loops}. 

A Polyakov loop is the trace of the ordered product of all time-like links ($U_\mu(t, \vec{x})$ with $\mu = t$) 
with the same space coordinates:

\begin{equation} \label{eq:polyloop_def}
	\phi(\vec{x}) = \Tr \prod_{\tau = 1}^{N_t} U_t(\tau, \vec{x}),
\end{equation}

where $\vec{x}$ represents the space coordinate of the loop (a $d$ dimensional vector, in the case of a $d+1$ dimensional lattice).

As usual, the correlation function is defined as

\begin{equation}
	\expval{\phi(\vec{x}) \phi(\vec{y})} = \frac{1}{\mathcal{Z}} \int \mathcal{D}U \phi(\vec{x}) \phi(\vec{y}) e^{-S[U]}.
\end{equation}

Now, the finite temperature interquark potential $V(R, T)$ can be extracted from the relation

\begin{equation}
	\expval{\phi(\vec{x}) \phi(\vec{y})} = e^{-\frac{1}{T} V(R, T)} = e^{-N_t V(R, T)},
\end{equation}

where the lattice spacing is set to $a = 1$ for simplicity and $|\vec{x} - \vec{y}| = R$.

Assuming an area law as with the Wilson loops, we have

\begin{equation} \label{eq:corr_fun_polyloops}
	\expval{\phi(\vec{x}) \phi(\vec{y})}_{N_t} \sim e^{-\sigma(T) N_t R},
\end{equation}

where $\sigma(T)$ is the \textit{finite temperature string tension}. Again, refer to \cite{gattringer} for a more
extensive overview about Polyakov loops.

\subsection{Polyakov loop as order parameter}

A compactified time direction with periodic boundary conditions has as consequence the appearance of a new
global symmetry in action $S$: its symmetry group is the center of the gauge group. In the case of an SU(N) gauge
group, the center is $\mathbb{Z}_N$; in the case of a Sp(N) gauge group, the center is always $\mathbb{Z}_2$.

Applying a center symmetry transformation has the effect of multiplying all time-like links belonging to the same space-like slice (they have
the same coordinate $t$) by the same element $W_0$ belonging to the center of the gauge group:

\begin{equation}
	U_t(\tau, \vec{x}) \longrightarrow W_0 U_t(\tau, \vec{x}) \qc \forall \vec{x}
\end{equation}

for a fixed value of $\tau$, the time component. From \eqref{eq:polyloop_def}, this implies that a Polyakov loop $\phi(\vec{x})$ also 
transforms in the same way:

\begin{equation}
	\phi(\vec{x}) \longrightarrow W_0 \phi(\vec{x}).
\end{equation}

Thus, when $\expval{\phi(\vec{x})} = 0$, the center symmetry is \textit{unbroken}; when $\expval{\phi(\vec{x})} \neq 0$, the center symmetry
is \textit{broken}. This means that the expectation value of the Polyakov loop is the order parameter of the spontaneous center symmetry
breaking phase transition.

Polyakov loops have another interesting property: their expectation value is related to the free energy $F$ of a single, isolated quark:

\begin{equation}
	\expval{\phi(\vec{x})} \sim e^{-\flatfrac{F}{T}}.
\end{equation}

We can then distinguish between confinement and deconfinement phase thanks to the expectation value of the Polyakov loop. A confined
phase is characterized by $F = \infty$, which implies $\expval{\phi(\vec{x})} = 0$; a deconfined phase is, on the contrary, characterized
by a finite $F$, thus $\expval{\phi(\vec{x})} \neq 0$. Not only is the Polyakov loop the order parameter of the center symmetry breaking phase
transition, but it is also the order parameter of the confinement-deconfinement phase transition. In the confined phase, the center symmetry
is unbroken, while is it spontaneously broken in the deconfined phase. 

As discussed in \cite{pepe}, while this is true in the infinite volume case, in a finite size scale analysis with a finite volume, 
the expectation value of a Polyakov loop is always zero: no spontaneous symmetry breaking occurs. We thus use the expectation
value of the magnitude of the Polyakov loop $\expval{|\phi(\vec{x})|}$ as order parameter. This quantity is always non-zero in a finite
volume $V$, but in the confined phase, for $V \rightarrow \infty$, we have $\expval{|\phi(\vec{x})|} = 0$.

Recalling the role that $\beta$ has in the LGT action \eqref{eq:lgt_action}, we can define a critical value of $\beta$, 
$\beta_c(N_t)$, which depends on $N_t$ and corresponds to the point at which a phase transition occurs. Inverting $\beta_c(N_t)$ and
using \eqref{eq:lattice_temperature}, we can define a critical temperature $T_c$:

\begin{equation}
	T_c = \frac{1}{N_{t,c}(\beta)},
\end{equation}

which represents the temperature at which the system undergoes a phase transition.

\subsection{The Svetitsky–Yaffe conjecture}

\subsection{Sp(N) group}

\section{Simulation and algorithm}
\red{Probably remove this intro because theory is described above}

Define the partition function of the $(2+1)$ dimensional system to be

\begin{equation}
	\mathcal{Z} = \int \mathcal{D}U e^{-\beta S(U)} \qq{where} \int \mathcal{D} U \equiv \prod_{x,\mu} \int \dd{U_{x,\mu}}.
\end{equation}

$U_{x,\mu}$ is an element of the group (a link variable starting at the lattice site $x$, pointing
along the direction $\mu$), 
and $S$ is the action, defined as
\begin{equation}
	S = \sum_\square S_\square \qc 
	S_\square \equiv - \frac{1}{4} \Tr(U_{x,\mu}U_{x+\hat{\mu}, \nu}U^\dagger_{x+\hat{\nu}, \mu}U^\dagger_{x,\nu}).
\end{equation}
The symbol $\square$ represents the plaquette: the minimum loop possible on the lattice. Note that a link 
variable has a direction: the adjoint of a link variable is the link connecting the two sites in the opposite
direction. In other words:
\begin{equation}
	U_{x,\mu} = U_{x+\hat{\mu}, -\mu}^\dagger.
\end{equation}

Note that the lattice has periodic boundary conditions. In our case, the gauge group of choice is Sp(2).

\subsection{Heat-bath algorithm}

The heat-bath algorithm is used to calculate each step of the evolution of the system.
The basic idea is to generate a new link element $U$ with a Boltzmann probability distribution:

\begin{equation}
	P(U) = \frac{1}{\mathcal{Z}} e^{-\beta S(U)} \dd{U}
\end{equation}  

for each link variable in the system. Updating each link once represents a single step in the algorithm. 

While this is fairly easy for the SU(2) gauge group using Creutz's algorithm \cite{creutz}, there is no obvious
way to generalize it to other gauge groups, like SU(N) or Sp(N). Thus, to generate new links belonging to Sp(2), we use
a more general approach, as designed by Cabibbo and Marinari \cite{cabibbo}.  

We consider a set $F$ of SU(2) subgroups of the gauge group Sp(2). Given a Sp(2) element $U$ of the form
\begin{equation}
	U = \mqty(
		W_{11} & W_{12} & X_{11} & X_{12} \\
		W_{21} & W_{22} & X_{21} & X_{22} \\
		X^*_{22} & -X^*_{21} & W^*_{22} & -W^*_{21} \\
		-X^*_{12} & X^*_{11} & -W^*_{12} & W^*_{11}
	),
\end{equation}
where $W_{ij}, X_{kl} \in \mathbb{C}$ for $i,j,k,l = 1, 2$, 
we can construct four SU(2) subgroups, by extracting two complex numbers $t_1$ and $t_2$:
\begin{itemize}
	\item $\begin{cases} t_1 = W_{11} \\ t_2 = X_{12} \end{cases}$
	\item $\begin{cases} t_1 = W_{22} \\ t_2 = X_{21} \end{cases}$
	\item $\begin{cases} t_1 = W_{11} + W_{22} \\ t_2 = X_{11} - X_{22} \end{cases}$
	\item $\begin{cases} t_1 = W_{11} + W^*_{22} \\ t_2 = W_{12} - W^*_{21} \end{cases}$
\end{itemize}
We can then build an SU(2) group element as

\begin{equation} \label{eq:su2element}
	a_k = \mqty(t_1 & t_2 \\ -t_2^* & t_1^*) \qc k = 1, 2, 3, 4,
\end{equation}

where $k$ labels each subgroup. 
Each choice of $t_1$ and $t_2$ above gives a different SU(2) element belonging to a SU(2) subgroup of Sp(2).

We define $A_k$ to be an SU(2) element belonging to the $k$th SU(2) subgroup, embedded into Sp(2). For each subgroup in $F$,
the Sp(2) embedding is constructed as follows:
\begin{itemize}
	\item $A_1 = \mqty(
		t_1 & 0 & 0 & t_2 \\
		0 & 1 & 0 & 0 \\
		0 & 0 & 1 & 0 \\
		-t^*_2 & 0 & 0 & t^*_1 
		)$
	\item $A_2 = \mqty(
		1 & 0 & 0 & 0 \\
		0 & t_1 & t_2 & 0 \\
		0 & -t^*_2 & t^*_1 & 0 \\
		0 & 0 & 0 & 1 
		)$
	\item $A_3 = \mqty(
		t_1 & 0 & t_2 & 0 \\
		0 & t_1 & 0 & -t_2 \\
		-t_2^* & 0 & t^*_1 & 0 \\
		0 & t^*_2 & 0 & t^*_1 
		)$
	\item $A_4 = \mqty(
		t_1 & t_2 & 0 & 0 \\
		-t^*_2 & t^*_1 & 0 & 0 \\
		0 & 0 & t_1 & t_2  \\
		0 & 0 & -t^*_2 & t^*_1 
		)$
\end{itemize}

Generating each $A_k$ randomly, we define the new link $U'$ to be

\begin{equation}
	U' = A_4 A_3 A_2 A_1 U,
\end{equation}

because, in this case, the set $F$ contains four SU(2) subgroups.

It is proven \cite{cabibbo} that this algorithm leads to thermalization, if each $A_k$ is randomly distributed as

\begin{equation} \label{eq:probdist}
	P(A_k) = \dd{A_k} \frac{e^{-\beta S(A_k U_{k-1})}}{\mathcal{Z}_k(U_{k-1})},
\end{equation}

where $U_{n} \equiv A_n A_{n-1} \dots A_1 U$ with $U_0 = U$ and 

\begin{equation}
	\mathcal{Z}_k(U) = \int_{SU(2)_k} \dd{A} e^{-\beta S(AU)}.
\end{equation}


The reason for the decomposition into SU(2) subgroups is to efficiently generate $A_k$ according to 
\eqref{eq:probdist}. In fact, now that we are dealing with SU(2) elements, we can fall back to Creutz's
algorithm \cite{creutz} to generate each SU(2) element, embed it into Sp(2) as explained above, and left multiply
the original link $U$ by it. 

Focusing on a single link $U$ to update, we are interested only in the plaquettes that contain $U$. Defining
$\widetilde{U}_i$ to be one of the staples surrounding $U$ (an ordered product of the three links in the plaquettes
containing $U$, that are not the link $U$ itself), we have 

\begin{align}
	S(A_k U) &= -\frac{1}{4} \Tr(A_k U \sum_i \widetilde{U}_i) + \text{terms independent of } A_k \\
	&= -\frac{1}{4} \Tr(a_k u_k \sum_i \tilde{u}_k^i) + \text{terms independent of } a_k, 
\end{align}

where $a_k$, $u_k$ and $\tilde{u}_k$ are SU(2) elements corresponding to the $k$th
subgroup extracted from $A_k$, $U$ and $\widetilde{U}$, respectively. This implies that we want to generate
$a_k$ according to the distribution

\begin{equation}
	\dd{P(a_k)} \sim e^{\frac{1}{4}\beta\Tr(a_k u_k \sum_i \tilde{u}_k^i)} \dd{a_k}.
\end{equation}

We parametrize $a_k$ as

\begin{equation} \label{eq:su2parametrization}
	a_k = \alpha_0 \id + i \vec{\alpha} \cdot \vec{\sigma},
\end{equation}

where $\alpha_\mu \in \mathbb{R} \ \forall \mu = 1, 2, 3, 4$ with the constraint that 

\begin{equation}
	\alpha^2 \equiv \alpha_0^2 + |\vec{\alpha}|^2 = 1
\end{equation} 

and $\vec{\sigma} = \qty(\sigma_1, \sigma_2, \sigma_3)$ is the three-vector of $2 \times 2$ Pauli matrices. 

The SU(2) group measure is then
\begin{equation}
	\dd{a_k} = \frac{1}{2\pi^2} \delta(\alpha^2 - 1) \dd[4]{\alpha}.
\end{equation}

Since the sum of SU(2) elements is proportional to an SU(2) element, we write
\begin{equation}
	u_k \sum_i \tilde{u}_k^i = c \bar{u}_k \qc \bar{u}_k \in \SU(2)
\end{equation}
where 
\begin{equation}
	c = \det(u_k \sum_i \tilde{u}_k^i)^{\flatfrac{1}{2}}.
\end{equation}

The probability distribution for $a_k$ now becomes
\begin{equation}
	\dd{P(a_k)} \sim e^{\frac{1}{4}\beta\Tr(c a_k \bar{u}_k)} \dd{a_k}.
\end{equation}

The group measure is invariant under multiplication by another SU(2) element:
\begin{equation}
	\dd(b a_k) = \dd{a_k} \qfor b \in \SU(2),
\end{equation}

so we can write
\begin{equation}
	\dd{P(a_k \bar{u}_k^{-1})} \sim e^{\frac{1}{4}\beta c \Tr(a_k)} \dd{a_k} 
	= \frac{1}{2\pi^2} e^{\frac{\beta}{2}c \alpha_0} \delta(\alpha^2 - 1) \dd[4]{\alpha},
\end{equation}

because $\Tr(a_k) = 2\alpha_0$. Noting that $\delta(\alpha^2 - 1)\dd[4]{\alpha} = 
\frac{1}{2}(1 - \alpha_0^2)^{\flatfrac{1}{2}} \dd{\alpha_0}\dd{\Omega}$, we rewrite $\dd{P(a_k \bar{u}_k^{-1})}$ as

\begin{equation}
	\dd{P(a_k \bar{u}_k^{-1})} \sim 
	\frac{1}{2\pi^2} \frac{1}{2}(1 - \alpha_0^2)^{\flatfrac{1}{2}} e^{\frac{\beta}{2}c \alpha_0} \dd{\alpha_0}\dd{\Omega}
\end{equation}

with $\alpha_0 \in (-1,1)$ and $\dd{\Omega}$ is the differential solid angle of the three-vector $\vec{\alpha}$,
which is of length $(1 - \alpha_0^2)^{\flatfrac{1}{2}}$.

The problem is now about generating the four-vector $\alpha_\mu$ according to the distribution above, thus obtaining
$a_k \in \SU(2)$. Finally, we obtain $A_k$ by embedding $a_k \bar{u}_k^{-1} \in \SU(2)$ into Sp(2). Doing this for every 
SU(2) subgroup will yield the new link $U'$.

To generate $a_k$, we have to randomly generate $\alpha_0$ according to

\begin{equation} \label{eq:probdistalpha}
	P(\alpha_0) \sim (1 - \alpha_0^2)^{\flatfrac{1}{2}} e^{\frac{\beta}{2}c\alpha_0}.
\end{equation}

The algorithm is quite simple. We uniformly generate $x$ in the range 

\begin{equation}
	e^{-\beta c} < x < 1
\end{equation} 

and define a trial $\alpha_0$ distributed according to $e^{\frac{\beta}{2}c\alpha_0}$ as

\begin{equation}
	\alpha_0 = 1 + \frac{2}{\beta c}\ln{x}.
\end{equation}

To account for the term $\qty(1 - \alpha_0^2)^{\flatfrac{1}{2}}$ in \eqref{eq:probdistalpha}, we \textit{reject} this trial
$\alpha_0$ with probability $1 - \qty(1 - \alpha_0^2)^{\flatfrac{1}{2}}$, generating a new trial $\alpha_0$ if 
the rejection is successful. We keep doing this until a trial $\alpha_0$ is finally accepted. 

The unit vector $\vec{\alpha} = (\alpha_1, \alpha_2, \alpha_3)$ is constructed by uniformly generating 

\begin{equation}
	\begin{aligned}
		\phi &\in (0, 2\pi), \\
		y &\in (-1, 1)
	\end{aligned}
\end{equation}

and defining 

\begin{equation}
	\begin{aligned}
		\theta &\equiv \arccos(y), \\
		r &\equiv \qty(1 - \alpha_0)^{\flatfrac{1}{2}},
	\end{aligned}
\end{equation}

$\alpha_0$ being the trial random number that has been accepted according to \eqref{eq:probdist}. 

We finally have, for $\vec{\alpha} = (\alpha_1, \alpha_2, \alpha_3)$,

\begin{equation}
	\begin{cases}
		\alpha_1 = r \sin(\theta) \cos(\phi) \\
		\alpha_2 = r \sin(\theta) \sin(\phi) \\
		\alpha_3 = r \cos(\theta)
	\end{cases}.
\end{equation}

$a_k$ will then be constructed using \eqref{eq:su2parametrization}, and $A_k$ using the embeddings described
above, with $t_1$ and $t_2$ selected from $a_k$ as of \eqref{eq:su2element}.

\subsection{Overrelaxation}

The full algorithm implemented for the simulation also takes advantage of \textit{overrelaxation} \cite{montvay}.
Overrelaxation is used to counter critical slowing down in the vicinity of the phase transition: here,
the system's correlation length is so large that lattice updates of the order of the lattice spacing are
negligible. This causes the simulation to slow down considerably. 
Overrelaxation is a way of choosing new link elements to update the lattice with, which prevents critical
slowing down. 

The idea is to choose a new link element $U'$ "as far as possible" from $U$ (the original link), without actually changing the
action $S$ of the system. \cite{montvay} describes overrelaxation in the case of SU(2) gauge group. For more
complex groups, it is not as straightforward to generate new links. Again, we use Cabibbo and Marinari's idea:
the new link is chosen left multiplying the old link $U$ by randomly generated $A_k$ elements belonging
to Sp(2), which are in turn built embedding SU(2) elements into Sp(2). $k$ labels the SU(2) subgroups and runs
from 1 to 4, as for the heat-bath algorithm. 

Note that overrelaxation is applied at each simulation step along with heat-bath: in fact,
a single complete simulation step consists of one or more steps of overrelaxation and one step of heat-bath, where
a step is the update of the entire lattice exactly once. In our case, overrelaxation is applied three times
before each heat-bath step. This is possible because, as mentioned, overrelaxation doesn't actually change
the action of the system.

Consider $\widetilde{U}_i$ to be one staple surrounding the link $U$, and let us define

\begin{equation}
	R = \sum_{i = 1}^6 \widetilde{U}_i.
\end{equation}

Evidently, there are six staples surrounding each link in a $(2+1)$ dimensional lattice, thus $i = 1,\dots,6$.

\cite{montvay} defines the new link to be
\begin{equation}
	U' = V U^{-1} V,
\end{equation}

where 

\begin{equation}
	V \equiv \det(R)^{\flatfrac{1}{2}} R^{-1},
\end{equation}

the inverse of the projection of $R$ onto the SU(2) gauge group. In the case of the Sp(2) gauge group, 
we define 

\begin{equation} \label{eq:overrelaxation_Ak}
	A_k U_{k-1} = V U_{k-1}^{-1} V,
\end{equation}

instead, where 

\begin{equation}
	U_{n} \equiv A_n A_{n-1} \dots A_1 U \qq{with} U_0 = U.
\end{equation}

Following Cabibbo and Marinari's prescription, the new link will be

\begin{equation} \label{eq:overrelaxation_newlink}
	U' = A_4 A_3 A_2 A_1 U
\end{equation}

in the case of four SU(2) subgroups.

We then extract the SU(2) elements corresponding to the $k$th subgroup of Sp(2) 
from each term in \eqref{eq:overrelaxation_Ak}:

\begin{eqnarray}
	\begin{cases}
		A_k \in \Sp(2) &\longrightarrow a_k \in \SU(2) \\
		U_{k-1} \in \Sp(2) &\longrightarrow u_k \in \SU(2) \\
		\widetilde{U}_i \in \Sp(2) &\longrightarrow r_k^{(i)} \in \SU(2)
	\end{cases}.
\end{eqnarray}

We also define
\begin{equation}
	r_k \equiv \sum_i r_k^{(i)}.
\end{equation}

Using the fact that a sum of SU(2) elements is proportional to a SU(2) element, we can write

\begin{equation}
	u_k r_k = u_k \sum_i r_k^{(i)} = c \bar{u}_k
\end{equation}

with $c = \det(r_k)^{\flatfrac{1}{2}}$ and $\bar{u}_k \in \SU(2)$.

Rewriting \eqref{eq:overrelaxation_Ak} as

\begin{equation}
	A_k = V U^{-1} V U^{-1}
\end{equation}

and applying the equation to the SU(2) elements extracted above, we end up with

\begin{equation}
	\begin{aligned}
		a_k &= \det(r_k) r_k^{-1} u_k^{-1} r_k^{-1} u_k^{-1} \\
		&= \det(r_k) (u_k r_k)^{-1} (u_k r_k)^{-1} \\
		&= \det(r_k) (r_k u_k)^{-2} \\
		&= \det(r_k) (c \bar{u}_k)^{-2}.
	\end{aligned}
\end{equation}

Given that $c = \det(r_k)^{\flatfrac{1}{2}}$, we finally have
\begin{equation}
	a_k = \bar{u}_k^{-2} = \qty(\frac{1}{c} u_k \sum_i r_k^{(i)})^{-2}.
\end{equation}

Exactly as with the heat-bath algorithm, once we find $a_k$, we can embed it into Sp(2) according to the
subgroup it belongs to, which yields the corresponding $A_k$, and find the new link using \eqref{eq:overrelaxation_newlink}.

\section{Results}
\red{Put plots and results of the fit and simulations: fit of susceptibility peaks, fit of beta vs nt, etc.}

\printbibliography
% https://www.overleaf.com/learn/latex/Bibliography_management_in_LaTeX

\end{document}