\documentclass[reqno,11pt]{article}

\usepackage[utf8]{inputenc}
\usepackage[tbtags]{amsmath}
\usepackage[english]{babel}
\usepackage{bm}
\usepackage{bbm} % \mathbbm{1}
\usepackage{amsfonts}
\usepackage{microtype}
\usepackage{physics}
\usepackage{tensor}
\usepackage{slashed}
\usepackage{subfig}
\usepackage{mathtools}
\usepackage[subfigure]{tocloft}
\usepackage[parfill]{parskip}
\usepackage{multirow}
\usepackage{xcolor}

\numberwithin{equation}{section}

\renewcommand{\cftsecleader}{\cftdotfill{\cftdotsep}}

\newcommand{\red}[1]{\textbf{\textcolor{red}{#1}}}

\newcommand{\SU}{\mathrm{SU}}
\newcommand{\Sp}{\mathrm{Sp}}
\newcommand{\id}{\mathbbm{1}}

\begin{document}

\begin{otherlanguage}{italian}

\pagestyle{empty}
    \begin{center}
        \textbf{\huge{Sintesi della relazione per la prova finale}} 
    \end{center}
    \medskip
    \textbf{Studente:} Nicholas Pini \hfill \textbf{Relatore:} prof. Leonardo Giusti \\
    \textbf{Matricola:} 813484       \hfill \textbf{Correlatore:} prof. Michele Pepe \\
    \textbf{Corso di laurea:} Fisica (magistrale) \\
	\textbf{Data seduta di laurea:} 21/11/2022 \\
    \textbf{Telefono:} 345 6954331

\subsection*{La transizione di deconfinamento nella teoria di Yang-Mills con gruppo di gauge Sp(2) in 3 dimensioni e studio degli effetti di stringa}

La cromodinamica quantistica (QCD) viene spesso studiata in modo non perturbativo sfruttando la regolarizzazione su reticolo: 
si considera lo spaziotempo discretizzato, e gli elementi del gruppo di gauge, detti "link", connettono i vari punti del reticolo.
In questo contesto, si è molto studiato il fenomeno del confinamento, per il quale una coppia di quark interagenti è caratterizzata da un potenziale
d'interazione che cresce linearmente con la distanza. È possibile modellare il potenziale d'interazione come una stringa che connette
le due particelle interagenti, avente la possibilità di vibrare: questo è un modello effettivo detto Effective String Theory. 
Nonostante sia un'approssimazione, risulta particolarmente efficace nel descrivere la dinamica a lunghe distanze. È interessante studiare
questo modello a temperatura finita (che in teorie su reticolo si realizza rendendo la direzione temporale periodica): il parametro d'ordine
della transizione di deconfinamento è il loop di Polyakov, che risulta essere anche il parametro d'ordine della rottura di simmetria
del centro del gruppo di gauge che compare a temperature finite. A questo proposito esiste una congettura di Svetisky e Yaffe che mette in corrispondenza il modello effettivo di stringa e la transizione di fase
di confinamento-deconfinamento in $(d+1)$ dimensioni con il modello di Ising in $d$ dimensioni e la sua transizione di fase ordinata-disordinata:
secondo la congettura, questi due sistemi appartengono alla stessa classe di universalità, e condividono perciò gli stessi esponenti critici.
La descrizione effettiva di stringa, per quanto efficace, non è completamente compatibile con la congettura, come è possibile vedere
dalla differenza degli esponenti critici con cui il modello di stringa approccia il punto critico rispetto agli esponenti critici
del modello di Ising.

Lo scopo di questo lavoro è di simulare la teoria di Yang-Mills con gruppo di gauge Sp(2) su un reticolo 3D a temperatura finita 
e studiare il potenziale di interazione fra due quark statici (massa infinita) in temperature appena inferiori alla temperatura
critica di transizione di fase. La simulazione è effettuata sfruttando l'algoritmo Heat-Bath come descritto da Cabibbo e Marinari.
La scelta del gruppo Sp(2) è dovuta al ruolo che il centro del gruppo di gauge ha nella transizione di deconfinamento: per i gruppi
Sp(N), infatti, il centro è sempre $\mathbb{Z}_2$, e non dipende quindi dalle dimensioni del gruppo di gauge stesso. Lo studio
della congettura è fatto misurando i valori che assume il loop di Polyakov su reticolo e calcolandone la funzione di correlazione
a due punti. Questa è messa in corrispondenza con l'andamento del correlatore spin-spin a corte e lunghe distanze del modello di Ising 2D,
che, secondo la congettura, appartiene alla stessa classe di universalità del modello studiato.

\end{otherlanguage}
\subsection*{Deconfinement transition in 3D Yang-Mills theory with Sp(2) gauge group and study of string effects}

\red{TODO}

\end{document}