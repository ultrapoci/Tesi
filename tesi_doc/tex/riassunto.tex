\documentclass[reqno,11pt]{article}

\usepackage[utf8]{inputenc}
\usepackage[tbtags]{amsmath}
\usepackage[english]{babel}
\usepackage{bm}
\usepackage{bbm} % \mathbbm{1}
\usepackage{amsfonts}
\usepackage{microtype}
\usepackage{physics}
\usepackage{tensor}
\usepackage{slashed}
\usepackage{subfig}
\usepackage{mathtools}
\usepackage[subfigure]{tocloft}
\usepackage[parfill]{parskip}
\usepackage{multirow}
\usepackage{xcolor}
\usepackage{verbatim} % for multiline comments

\numberwithin{equation}{section}

\renewcommand{\cftsecleader}{\cftdotfill{\cftdotsep}}

\newcommand{\red}[1]{\textbf{\textcolor{red}{#1}}}

\newcommand{\SU}{\mathrm{SU}}
\newcommand{\Sp}{\mathrm{Sp}}
\newcommand{\id}{\mathbbm{1}}

\begin{otherlanguage}{italian}
\begin{comment}
Teorie di gauge non abeliane sono caratterizzate da fase confinante di bassa temp

Due cariche di colore quark statici interagiscono con potenziale linearmente con distanza

Linee di forza sono nel tubo di flusso -> EST (corda vibrante a basse energie)

Conseguenze su andamento del potenziale -> Luscher

Teoria a T finita -> transizione di deconfinamento -> trans di fase di 1 o 2 ordine

2 ordine -> universale -> caratteristiche indipendenti dalla teoria: dipende da dim spazio e simm
rotta -> simmetria centro -> centro Z2 -> secondo ordine modello di Ising 2D -> è la congettura
di Svetisky Yaffe -> prevede una lung di correlazione con esponente critico nu = 1

Si ricollega a EST -> sebbene sia valida in regime di bassa temperatura, prevede la presenza
di trans di fase a T finita con nu = 1/2 

Problema: cercare di capire se EST e congettura possono essere concigliati. Sono fenomeni
non perturbativi => regolarizzazione della teoria su reticolo => simulazioni monte carlo
=> info non perturbative

Occupato di studiare questi fenomeni nel caso Sp(2) con 2+1 dimensioni che ha trans di secondo
ordine => studio del comportamento critico e gli aspetti legati a EST 

Sviluppo di codice parallelo da zero => effettuato simulazione numeriche non perturbative 
per Nt = 5, 6, 7, 8 e L = 40,60,80,100 => misura del correlatore del loop di Polyakov 
a T critica (regime critico di stringa)
\end{comment}    
\end{otherlanguage}

\begin{document}

\begin{otherlanguage}{italian}

\pagestyle{empty}
    \begin{center}
        \textbf{\huge{Sintesi della relazione per la prova finale}} 
    \end{center}
    \medskip
    \textbf{Studente:} Nicholas Pini \hfill \textbf{Relatore:} prof. Leonardo Giusti \\
    \textbf{Matricola:} 813484       \hfill \textbf{Correlatore:} prof. Michele Pepe \\
    \textbf{Corso di laurea:} Fisica (magistrale) \\
	\textbf{Data seduta di laurea:} 21/11/2022 \\
    \textbf{Telefono:} 345 6954331

\subsection*{La transizione di deconfinamento nella teoria di Yang-Mills con gruppo di gauge Sp(2) in 3 dimensioni e studio degli effetti di stringa}
Le teorie di gauge non abeliane sono caratterizzate dal fenomeno del confinamento di colore: sotto una certa temperatura 
critica -- nella fase detta ”confinante” -- due cariche di colore in rappresentazione fondamentale (quark) interagiscono con un 
potenziale che cresce asintoticamente in modo lineare con la distanza. Ciò fa sì che una carica di colore isolata non sia 
direttamente osservabile. Nonostante non ci sia una dimostrazione analitica del fenomeno del confinamento, ne esistono evidenze 
tramite calcoli numerici.

Le linee di forza tra le due cariche di colore sono focalizzate in un tubo di flusso che può essere descritto come una stringa 
vibrante. Questo modello è detto Effective String Theory (EST) e rappresenta una teoria effettiva di bassa energia per 
l’interazione tra quark statici, valida a grandi distanze $R$ e basse temperature. Esso risulta essere efficace e predittivo: 
ad esempio, il primo termine dell’espansione a lunghe distanze di questo modello ha come conseguenza la presenza di un termine 
correttivo proporzionale a $\flatfrac{1}{R}$ nel potenziale d’interazione fra quark, detto termine di L{\"u}scher. Tale termine
ha un coefficiente numerico ben definito ed ha avuto molti riscontri in simulazioni di varie teorie di gauge su reticolo.

Il fenomeno del confinamento dipende dalla temperatura della teoria: a una data temperatura critica il sistema passa dalla fase 
confinata (bassa temperatura) ad una fase deconfinata (alta temperatura), nella quale il tubo di flusso viene dissolto dalle 
fluttuazioni di stringa e il potenziale fra le due cariche di colore non è più confinante ma schermato. Estrapolando la 
descrizione effettiva di stringa alla temperatura critica, essa prevede una transizione di fase di deconfinamento di secondo 
ordine con esponente critico $\nu = \flatfrac{1}{2}$ per la lunghezza di correlazione. 
In generale, le transizioni di fase possono essere di primo o secondo ordine: per quelle di secondo ordine, la lunghezza di correlazione 
diverge al punto critico. A ciò è associato il concetto particolarmente importante di classe di universalità, per il quale 
sistemi fisici molto diversi fra loro, aventi però la stessa dimensionalità e la stessa simmetria, sono descritti dalla stessa 
teoria nell’intorno della transizione di fase. In particolare, gli esponenti che caratterizzano il comportamento critico di 
osservabili corrispondenti sono gli stessi nelle diverse teorie.

Per quanto riguarda le teorie di gauge non abeliane, alla transizione di deconfinamento avviene la rottura spontanea della 
simmetria del centro del gruppo di gauge. Esiste una congettura dovuta a Svetitsky e Yaffe, secondo cui, considerando ad esempio 
una teoria di gauge $(d+1)$ dimensionale con centro del gruppo di gauge $\mathbb{Z}_2$, se la transizione di fase di deconfinamento
è di secondo ordine, allora il sistema è nella stessa classe di universalità del modello di Ising $d$ dimensionale. Questo 
significa che l’esponente critico $\nu$ associato alla lunghezza di correlazione della teoria di gauge in $(2+1)$ dimensioni è lo 
stesso del modello di Ising in 2 dimensioni, il cui valore è ben conosciuto: $\nu = 1$. Questo risultato però è in contrasto 
con l’analisi fatta usando la EST, che prevede $\nu = \flatfrac{1}{2}$. La teoria effettiva di stringa è molto accurata a 
bassa temperatura mentre il regime critico è al limite del suo intervallo di applicabilità: risulta dunque interessante 
studiare in modo quantitativo come si passa dal regime di validità della EST a quello critico regolato dalla congettura 
di Svetistky e Yaffe.

Dato che entrambi i regimi sono fortemente non perturbativi è utile considerare la formulazione della teoria di gauge non 
abeliana su reticolo. Essa costituisce una regolarizzazione nella quale si discretizza lo spaziotempo su un reticolo rendendo
la teoria e il path integral ben definiti. Considerando tale formulazione, è possibile effettuare lo studio 
non perturbativo mediante simulazioni numeriche su un calcolatore. La teoria può essere studiata anche a temperatura finita 
imponendo condizioni periodiche per la direzione temporale, con la temperatura data dall’inverso della lunghezza della 
dimensione temporale.

Nel lavoro di tesi mi sono occupato di studiare la transizione di fase di deconfinamento e gli effetti di stringa nella teoria 
di Yang-Mills in $(2+1)$ dimensioni con gruppo di gauge $\Sp(2)$. La scelta di questo gruppo è dovuta al fatto che la transizione 
di deconfimento è del secondo ordine, ed inoltre i gruppi $\Sp(N)$ hanno $\mathbb{Z}_2$ come centro del gruppo per ogni valore di 
$N$: la simmetria rotta quindi non dipende dalla dimensionalità del gruppo di gauge stesso, come invece avviene coi gruppi 
$\SU(N)$ più comunemente scelti. Si è sviluppato da zero un codice parallelo che permette di simulare la teoria di gauge tramite
metodi Monte Carlo: in particolare, per l’aggiornamento delle configurazioni del campo di gauge, si è implementato l’algoritmo 
Heat-Bath come sviluppato da Cabibbo e Marinari insieme con il metodo dell’over-relaxation per accelerare la decorrelazione 
delle configurazioni che vengono progressivamente generate. Sono state effettuate simulazioni per valori della lunghezza del 
reticolo in direzione temporale $N_t = 5, 6, 7, 8$ e in direzione spaziale $N_s = 40, 60, 80, 100$. Lo studio della transizione di 
fase di deconfinamento è avvenuto mediante l’analisi di Finite Size Scaling del parametro d’ordine della rottura di simmetria 
(il loop di Polyakov) e della sua suscettività. 
Infine, in fase confinata a temperatura prossima alla transizione di deconfinamento si è studiata l’interazione tra due cariche 
di colore statiche in rappresentazione fondamentale misurando con simulazioni Monte Carlo la funzione a due punti 
del loop di Polyakov. Si è osservato che i risultati numerici sono ben descritti dalla EST a grandi distanze mentre il 
comportamento a distanze dell’ordine della lunghezza di correlazione o inferiori, è ben rappresentato dal regime critico del 
modello di Ising in $d = 2$. 

\end{otherlanguage}
\subsection*{Deconfinement transition in 3D Yang-Mills theory with Sp(2) gauge group and study of string effects}

Non abelian gauge theories exhibit color confinement, a phenomenon by which, under a certain critical temperature,
two color charges in the fundamental representation (quarks) interact through an asymptotically linear potential which
increases with distance: this is the "confined" phase. Because of color confinement, a single, isolated color charge
is not directly observable. Despite numerical evidence of color confinement, an analytical proof of this phenomenon has
yet to be found. 

The field lines between the two color charges form a flux tube, which can be described as a vibrating string. This model
is called Effective String Theory (EST) and it is a low energy effective theory for interacting static quarks, valid at
low temperatures and large distances $R$. EST is highly predictive: indeed, the first term of the large distance expansion
of this model predicts a term proportional to $\flatfrac{1}{R}$ present in the interquark potential, called L{\"u}scher term.
The L{\"u}scher term has a well defined coefficient and it is in agreement with many simulated gauge theories on the lattice.

Color confinement depends on the temperature of the theory: at a critical temperature, the system changes from the
confined phase (low temperature) to the deconfined phase (high temperature). In the latter phase, the flux tube is disrupted
by string fluctuations and the potential between the two color charges is screened and no longer confining. The string 
description at the critical temperature predicts a second order phase transition with a critical exponent 
$\nu = \flatfrac{1}{2}$ for the correlation length. In general, phase transitions can be of first or second order. In second
order phase transitions, the correlation length diverges at the critical point. The concept of universality classes is
of particular importance in studying these transitions: very different systems, sharing the same dimensionality and
symmetry, are described by the same theory in the vicinity of the critical point. This is reflected by the critical exponents,
which describe how some observables approach the critical point and are the same for theories belonging 
to the same universality class.

In non abelian gauge theories, the spontaneous symmetry breaking of the center of the gauge group is connected
to the deconfinement transition. A conjecture by Svetisky and Yaffe states that a $(d+1)$ dimensional gauge theory with a
second order deconfinement transition and $\mathbb{Z}_2$ as center of the gauge group is in the same universality class
containing the $d$ dimensional Ising model. In particular, this implies that the critical exponent $\nu$ associated to the
correlation length of a $(2+1)$ dimensional gauge theory is $\nu = 1$, the same critical exponent found in the 2 dimensional
Ising model. This is contrast with the EST result of $\nu = \flatfrac{1}{2}$. At low temperatures, the string description
is quite accurate, but it starts to become unreliable in the vicinity of the critical point: it is interesting to 
quantitatively study how the system passes from the regime of valid EST description to the critical regime described by
Svetisky and Yaffe's conjecture.

It is useful to consider the non abelian gauge theory regularized on a lattice, considering the non perturbative nature
of both the regimes. This regularization discretizes spacetime on a lattice, which causes the theory and its path integral
to be well defined. This makes it possible to simulate the theory on a computer. The finite temperature theory is achieved
by imposing periodic boundary conditions for the lattice's temporal direction, whose length is now interpreted
as the inverse of the temperature of the system. 

In making this thesis, I've studied the deconfined phase transition and string effects of the 3 dimensional Yang-Mills theory 
with $\Sp(2)$ gauge group. The choice of this gauge group has been driven by the fact that the deconfinement phase transition
is second order, and the center of $\Sp(N)$ gauge groups is $\mathbb{Z}_2$ for every value of $N$. 
Thus, the spontaneous symmetry breaking does not depend on the dimensionality of the gauge group,
in contrast with the more common $\SU(N)$ gauge groups. I've developed a parallel program from scratch in order to simulate
the gauge theory using Monte Carlo methods: in particular, the Heat-Bath algorithm, as designed by Cabibbo and Marinari, has been
used to update the gauge field's configurations, together with the overrelaxation algorithm which speeds up the
decorrelation of the generated lattice configurations. Simulations where made for lattices of dimensions 
$N_t = 5, 6, 7, 8$ and $N_s = 40, 60, 80, 100$. The deconfinement phase transition has been studied through
the Finite Size Scaling analysis of the order parameter of the broken symmetry (the Polyakov loop) and its susceptibility. 
Finally, the interaction of the two static color charges in fundamental representation has been studied in the
confined phase slightly below the critical temperature of phase transition, by measuring the two point function of the Polyakov
loop using Monte Carlo simulations. Numerical results of these simulations are well described by the EST for large
distances. For distances below or near the correlation length, the behavior of the system is well described by the
2D Ising model in the critical regime.    


\end{document}