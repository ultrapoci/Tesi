\documentclass[reqno,11pt]{article}

\usepackage[utf8]{inputenc}
\usepackage[tbtags]{amsmath}
\usepackage[english]{babel}
\usepackage{bm}
\usepackage{bbm} % \mathbbm{1}
\usepackage{amsfonts}
\usepackage{microtype}
\usepackage{physics}
\usepackage{tensor}
\usepackage{slashed}
\usepackage{subfig}
\usepackage{mathtools}
\usepackage[subfigure]{tocloft}
\usepackage[parfill]{parskip}
\usepackage{multirow}
\usepackage{xcolor}
\usepackage{verbatim} % for multiline comments

\numberwithin{equation}{section}

\renewcommand{\cftsecleader}{\cftdotfill{\cftdotsep}}

\newcommand{\red}[1]{\textbf{\textcolor{red}{#1}}}

\newcommand{\SU}{\mathrm{SU}}
\newcommand{\Sp}{\mathrm{Sp}}
\newcommand{\id}{\mathbbm{1}}

\begin{otherlanguage}{italian}
\begin{comment}
Teorie di gauge non abeliane sono caratterizzate da fase confinante di bassa temp

Due cariche di colore quark statici interagiscono con potenziale linearmente con distanza

Linee di forza sono nel tubo di flusso -> EST (corda vibrante a basse energie)

Conseguenze su andamento del potenziale -> Luscher

Teoria a T finita -> transizione di deconfinamento -> trans di fase di 1 o 2 ordine

2 ordine -> universale -> caratteristiche indipendenti dalla teoria: dipende da dim spazio e simm
rotta -> simmetria centro -> centro Z2 -> secondo ordine modello di Ising 2D -> è la congettura
di Svetisky Yaffe -> prevede una lung di correlazione con esponente critico nu = 1

Si ricollega a EST -> sebbene sia valida in regime di bassa temperatura, prevede la presenza
di trans di fase a T finita con nu = 1/2 

Problema: cercare di capire se EST e congettura possono essere concigliati. Sono fenomeni
non perturbativi => regolarizzazione della teoria su reticolo => simulazioni monte carlo
=> info non perturbative

Occupato di studiare questi fenomeni nel caso Sp(2) con 2+1 dimensioni che ha trans di secondo
ordine => studio del comportamento critico e gli aspetti legati a EST 

Sviluppo di codice parallelo da zero => effettuato simulazione numeriche non perturbative 
per Nt = 5, 6, 7, 8 e L = 40,60,80,100 => misura del correlatore del loop di Polyakov 
a T critica (regime critico di stringa)
\end{comment}    
\end{otherlanguage}

\begin{document}

\begin{otherlanguage}{italian}

\pagestyle{empty}
    \begin{center}
        \textbf{\huge{Sintesi della relazione per la prova finale}} 
    \end{center}
    \medskip
    \textbf{Studente:} Nicholas Pini \hfill \textbf{Relatore:} prof. Leonardo Giusti \\
    \textbf{Matricola:} 813484       \hfill \textbf{Correlatore:} prof. Michele Pepe \\
    \textbf{Corso di laurea:} Fisica (magistrale) \\
	\textbf{Data seduta di laurea:} 21/11/2022 \\
    \textbf{Telefono:} 345 6954331

\subsection*{La transizione di deconfinamento nella teoria di Yang-Mills con gruppo di gauge Sp(2) in 3 dimensioni e studio degli effetti di stringa}

Le teorie di gauge non abeliane sono caratterizzate dal fenomeno del confinamento di colore: al di sotto di una certa
temperatura critica, quindi nella fase detta "confinante", due cariche di colore (quark) interagiscono con un potenziale che
cresce linearmente con la distanza, nel limite di grandi distanze. Questo fa sì che una carica di colore non è direttamente
osservabile singolarmente. Nonostante non ci sia ancora una rigorosa dimostrazione del fenomeno del confinamento, ne esistono
varie prove sperimentali. 

Il potenziale di interazione fra due quark genera un tubo di flusso che connette le due cariche: ciò
suggerisce di modellare questo potenziale come una sottile stringa vibrante che collega i due quark. Questo modello
è detto Effective String Theory (EST) e, come suggerisce il nome, è solo un modello effettivo valido a grandi distanze e
non una completa descrizione non perturbativa della teoria. Ciò nonostante, è un modello estremamente efficace e predittivo: 
già solo il primo termine dell'espansione a lunghe distanze di questo modello ha come conseguenza la presenza di un termine
correttivo nel potenziale d'interazione fra quark, detto termine di L{\"u}scher, il quale ha avuto molti riscontri 
in simulazioni della teoria su reticolo. 

Il fenomeno del confinamento dipende dalla temperatura della teoria, e comporta una transizione di fase: a una certa 
temperatura critica, il sistema passa dalla fase confinata (bassa temperatura) ad una fase deconfinata (alta
temperatura), nella quale il tubo di flusso viene distrutto dalle fluttuazioni di stringa e non c'è più presenza
di potenziale confinante fra le due cariche di colore. Di particolare interesse nell'ambito delle transizioni di fase
è lo studio degli esponenti critici: descrivono l'andamento di quantità fondamentali del sistema nell'intorno della 
temperatura critica. A questo proposito, una di queste quantità è la lunghezza di correlazione del sistema, e la descrizione
effettiva di stringa prevede che l'esponente critico associato a tale quantità è $\nu = \flatfrac{1}{2}$. In generale,
le transizioni di fase possono essere di 1° o 2° ordine: nel secondo caso, è particolarmente importante il concetto
di classi di universalità, per il quale sistemi fisici molto diversi fra loro, aventi però la stessa dimensionalità e
rottura di simmetria, sono descritti dagli stessi esponenti critici. Per quanto riguarda il confinamento di cariche di
colore, esiste una congettura dovuta a Svetisky e Yaffe: in un sistema $(d+1)$ dimensionale, se la transizione 
di fase di deconfinamento è di secondo ordine, allora il sistema è nella stessa classe di universalità del modello di
Ising $d$ dimensionale: in entrambi i sistemi, la simmetria rotta è una simmetria globale associata al centro del gruppo
di gauge della teoria originale. Questo significa che l'esponente critico $\nu$ associato alla lunghezza di
correlazione della teoria di gauge in $(2+1)$ dimensioni è lo stesso del modello di Ising in 2 dimensioni, cioè
$\nu = 1$. Questo risultato però è in contrasto con l'analisi fatta usando la EST, che prevede $\nu = \flatfrac{1}{2}$.

L'obiettivo quindi è quello di studiare teorie di gauge non abeliane e verificare se la discrepanza fra EST e la
congettura di Svetisky e Yaffe può essere risolta. Trattandosi di fenomeni non perturbativi, si ricorre alla Lattice Gauge
Theory (LGT) per studiarli: si tratta di una regolarizzazione della teoria per la quale si discretizza lo spaziotempo
su un reticolo a passo reticolare costante. Ciò rende la teoria e il path integral ben definiti, e, simulando la teoria
su un calcolatore, permette di ottenere informazioni non perturbative sul sistema e le sue osservabili. La teoria è definita
con temperatura finita rendendo la direzione temporale periodica. 

Questo lavoro si occupa di studiare il fenomeno del confinamento nella teoria di Yang-Mills in $(2+1)$ dimensioni con 
gruppo di gauge Sp(2): la scelta di questo gruppo è dovuta al fatto che la transizione di deconfimento è ben risaputo
essere del secondo ordine, ed inoltre i gruppi Sp(N) hanno $\mathbb{Z}_2$ come centro del gruppo per ogni valore di $N$, 
perciò la simmetria rotta non dipende dalla dimensionalità del gruppo di gauge stesso (come invece avviene coi gruppi
SU(N) più comunemente scelti). Si è sviluppato un codice su calcolatore da zero che permette di simulare la LGT tramite
metodi Monte Carlo: in particolare, si è utilizzato l'algoritmo Heat-Bath come sviluppato da Cabibbo e Marinari per la 
generazione di nuove configurazioni Monte Carlo del sistema con gruppo di gauge Sp(2). Sono state effettuate simulazioni
per valori della lunghezza del reticolo in direzione temporale $N_t = 5, 6, 7, 8$ e in direzione spaziale 
$L = 40, 60, 80, 100$, con l'obiettivo di studiare la transizione di fase e ridurre effetti di finite size scaling dovuti a volumi finiti. È stato 
inoltre misurato il loop di Polyakov (il parametro d'ordine della transizione di fase a cui siamo interessati) 
a temperature appena sotto la temperatura critica di transizione di fase di deconfinamento, con l'obiettivo 
di studiarne il correlatore, a cui è associato il potenziale di interazione fra le due cariche di colore, e gli effetti
di stringa.

\end{otherlanguage}
\subsection*{Deconfinement transition in 3D Yang-Mills theory with Sp(2) gauge group and study of string effects}

\red{TODO}

\end{document}